\chapter*{Calculation of the ChiSquare ($\chi^2$)}
\setcounter{chapter}{1}
\emph{Get the ChiSquare}

\section{Introduction}
The purpose of the \emph{ChiSquare} plugin is to calculate the ChiSquare and the reduced ChiSquare for two sets of data.
 
\section{Plugin Properties}
Table \ref{table:PluginProperties} lists available plugin property names, along with their data type and purpose.


\begin{table}[ht]
\centering % used for centering table
\begin{tabular}{l l p{7.5cm}} % centered columns (4 columns)

Parameter Name & Data Type & Purpose \\ [0.5ex] % inserts table 
%heading
\hline % inserts single horizontal line
ExperimentalData		& 	TelluriumData   	& Data representing Experimental data. \\
ModelData 				& 	TelluriumData   	& Data representing Model data. \\
NrOfModelParameters		& 	int   				& Number of model parameters used to create the model data. \\
ChiSquare     			& 	double		   		& The calculated ChiSquare. \\
ReducedChiSquare 		& 	double		   		& The calculated reduced ChiSquare. \\

\hline %inserts single line
\end{tabular}
\caption{Plugin Properties} 
\label{table:PluginProperties} 
\end{table}

\section{Plugin Events}
This plugin does not use any plugin events.

\section{The \texttt{execute()} function}
The \verb|execute()| function will attempt to calculate the ChiSquare, and the reduced ChiSquare, using data supplied by the user. 

\section{Python examples}

\subsection{Usage of the ChiSquare plugin}
The python script below shows how to use the plugin. 

\begin{singlespace}
\lstinputlisting[label=chisquare_plugin_header,caption={ChiSquare plugin example.},language=Python]{Examples/telChiSquare.py}
\end{singlespace}

